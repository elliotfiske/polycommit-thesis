\chapter{Background}
\section{Early Computing}

\subsection{Pressey's Teaching Machines}

\par The concept of ``Learning Machines'' are actually older than Turing machines\footnote{We use the definition of ``Learning Machines'' provided in \cite{benjamin1988history}. That is, a teaching machine is an automated device that provides questions to the user and automatically grades them.}. The earliest learning machine was patented by Sidney L. Pressey in 1928 (See \textbf{\hyperref[fig:pressey_machine]{Figure \ref*{fig:pressey_machine}}}) \cite{benjamin1988history}. This machine was designed for both \textit{testing} and \textit{teaching}. In \textit{testing} mode, it would only give the user one chance to answer each question, but in \textit{teaching} mode the user could try as many times as they liked. After answering a specified number of questions correctly, the machine would dispense a piece of candy. The machine came equipped with a ``reward dial'' that allowed the researcher to define how many questions the user had to respond to before they earned the candy.

\par Unfortunately, Pressey was not able to find a market for his teaching machines. Skinner would later argue that it was the general culture of the 1920s and 30s that stopped Pressey's machines from gaining popularity; it wasn't until the 1950s and 60s where novel technology was more readily accepted into the mainstream \cite{skinner1958teaching}.

 \begin{figure}[h]
 	\includegraphics[width=1.0\linewidth]{figures/pressey_machine}
 	\caption{A diagram depicting the patent for Sidney Pressey's teaching machine. Candy was dispensed to the user after answering a question correctly.}
 	\label{fig:pressey_machine}
 	\cite{benjamin1988history}
 \end{figure}
 
 \subsection{Skinner's Teaching Machines}
 \par Dr. B. F. Skinner continued the work of Pressey. In 1958, he published a paper \cite{Skinner969} describing his newer teaching machine based off punchcard computing technology. While using Skinner's machine, students receive a rotation of questions about a certain subject, and could not continue until they had answered each question in the rotation correctly. The cycle of questions presented the information in varied ways. For instance, in \textbf{\hyperref[fig:skinner_machine]{Figure \ref*{fig:skinner_machine}}}, a series of questions that teach the student how to spell ``manufacture'' is displayed. The first question simply asked the student to repeat the word, and the rest of the questions transform that question in various ways, which Skinner hoped would ``hold the student's attention'' and force them to process the information in different ways, accelerating the learning process. 
 
 \begin{figure}[h!]
 	\includegraphics[width=1.0\linewidth]{figures/skinner_machine}
 	\caption{A series of questions as presented by Skinner's teaching machine. The question is ``transformed'' to several different modes that help the user learn the information faster. }
 	\label{fig:skinner_machine}
 	\cite{benjamin1988history}
 \end{figure}
 

\par Dr. Skinner believed that computers would completely revolutionize classrooms. Students would be able to learn at their own pace, with the computer determining the optimal schedule for each student based on how they were performing. Skinner made it clear that he didn't envision his computers as replacing human teachers, but rather supplementing their teaching, saving them time and making learning a more efficient process.

\subsection{Behavior and Learning}

Of course, B. F. Skinner is most famous for his creation of the ``Skinner Box,'' \cite{skinner1963operant} a device that gave simple reinforcement to animals or humans in the lab in order to slowly shape their behavior, through a process known as operant conditioning. When a subject performed well according to the test, they would receive a small reward, while if they strayed from the purpose of the test, they would not receive a reward. Dr. Skinner argued that ``much of what we know [about the learning process] has come from studying the behavior of lower organisms, [and] the results hold surprisingly well for human subjects'' \cite{skinner1958teaching}. Skinner's research hearkens back to the candy that was dispensed by the early teaching machines of Pressey.

\par It's important to note the connection between operant condition and learning. Skinner believed that learning was simply a form of behavior modification, and that students could learn far more effectively by using the techniques Skinner had discovered during his work on operant conditioning.

\subsubsection{Rewards Scheduling}
 One of the ways that Skinner discovered that operant conditioning could be accelerated was through \textit{variable ratio reinforcement scheduling}. \cite{ferster1957schedules}. There are several different schedules of reinforcement a researcher can use (See \textbf{\hyperref[fig:variable_ratio]{Figure \ref*{fig:variable_ratio}}}).
 
 \par The classical method is \textbf{continuous reinforcement}, where the reward is given for every single instance of the desired action. 
 
 \par \textbf{Fixed-ratio scheduled reinforcement} is where a reward is given after a set amount of instances of the desired behavior. This is similar to how Pressey's machine dispensed candy after a set amount of correct answers. 
 
 \par \textbf{Fixed-interval scheduled reinforcement} is where a reward is dispensed for the first instance of a desired behavior \textit{in a given time frame.} That is, if the time frame is 1 minute, a reward will be given on the first bar-press that occurs 1 minute or later than the last reward. This is the rewards schedule used by \textit{Polycommit}.
 
 \par Finally, the schedule that drives the most response is \textbf{variable ratio rewards scheduling}. Using this method, a reward is given after a random number of instances of the desired behavior. This scheduling is most engaging for the test subject, and causes them to be conditioned much faster. However, we decided this was not an appropriate scheduling system to use for \textit{Polycommit}, as it would create an inconsistent experience for the user.
 
 
 \begin{figure}[h]
 	\includegraphics[width=1.0\linewidth]{figures/variable_ratio}
 	\caption{Variable ratio rewards scheduling. Note that as rewards are spaced out randomly, the desired behavior appears much more quickly.}
 	\label{fig:variable_ratio}
 \end{figure}

\section{Habit Formation}
\par Recent studies in behavior modification show that the most effective way to bring about a long-term change in behavior to an individual's life is to form habits. BJ Fogg, a Stanford researcher and a TED presenter, has done extensive research into how humans form their behavior patterns. In one of his papers, he breaks down the formation of new behaviors into a model that depends on what he calls ``triggers.'' \cite{fogg2009behavior} While behavior \textit{can} be modified long-term given enough motivation or ability, the most effective change in behavior comes from low-effort, routine tasks. As the user's comfort level grows, their willingness to change their behavior also increases.

\par BJ Fogg ties these ideas directly into computer interface design in his most prominent paper \cite{2002persuasive}. He presents the idea that user interfaces should convey a comforting, friendly tone to increase the user's comfort, and thus lower the barrier to creating new habits. He argues that human-computer interaction is surprisingly similar to human-human interaction. He showed several early examples of ``assistant'' computer applications, noting how they raise an emotional response, although the early technology of the time limited their effectiveness. It brings to mind Microsoft's infamous Clippy character, which certainly made an impact in user's lives, perhaps for being more annoying than helpful. Interesting to note is that this paper predates the release of Apple's Siri personal assistant by one year.

\par Other studies back up Fogg's characterization of habit formation and behavior modification. In a 2010 study by Dr. Lally, they emphasize the importance of \textit{familiarity} to habit formation, and note that the easiest way to form habits is to tie them to an existing, automatic behavior. For instance, test subjects in the study by Dr. Lally determined that it was easiest to condition subjects to perform some action by instructing them to perform it right after eating breakfast in the morning. By using the powerful force behind an existing habit, it is possible to ``reprogram'' your behavior such that a new habit is formed with the new desired behavior.

\subsection{Potential Hazards of Gamification}
According to Dr. Renfree, the advent of ``apps'' and habit-forming application has created an essentially new field in psychology \cite{renfree2016don}. In his study, Dr. Renfree argues that the behavior modifications seen in apps like Lift and Memrise represent new advances into how habit-forming psychology operates.

\par However, he claims that most of these new effects are actually negative. For instance, he notes that oftentimes the new habits generated by these apps are ``fragile,'' since they are so dependent on a tight dopamine-based reinforcement loop. As soon as the loop is broken, the brain ``loses interest'' and the new information is deprioritized. As we attempt to bring gamification into the classroom, we need to ensure that we don't bombard the user with information and make their understnding ``fragile.''




% TODO: Why is this here? Shouldn't it be in related works
%\section{Duolingo}
%As mentioned previously in the introduction, Duolingo is an app that already uses several of the processes that we are describing. Duolingo has been shown to be very effective for learning new languages, even perhaps more effective than typical classes. However, the effectiveness of Duolingo is mostly contingent on the motivation of the student. If the student is going to a foreign country soon, Duolingo is the most effective since there is a significant time pressure and pressure to learn the language in order to fit in at whatever country the student is planning to go to \cite{vesselinov2012duolingo}. If, however, the student is simply learning another language for fun, there will be significantly less benefit to them. We must consider this as we develop our educational software; apparently the nature of the student's motivation and their reason for wanting to learn the subject material plays heavily into wheter or not they will be successful in using the app.